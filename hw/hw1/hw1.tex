\documentclass{article}
\usepackage[utf8]{inputenc}
\usepackage[margin=1in]{geometry}
\usepackage{enumitem}
\usepackage{amsmath}
\usepackage{listings}
\usepackage{color}
\usepackage{booktabs}
\usepackage[T1]{fontenc}

% macro to select a scaled-down version of Bera Mono (for instance)
\makeatletter
\newcommand\BeraMonottfamily{%
  \def\fvm@Scale{0.85}% scales the font down
  \fontfamily{fvm}\selectfont% selects the Bera Mono font
}
\makeatother

\definecolor{codegreen}{rgb}{0,0.6,0}
\definecolor{codegray}{rgb}{0.5,0.5,0.5}
\definecolor{codepurple}{rgb}{0.58,0,0.82}
\definecolor{backcolour}{rgb}{0.95,0.95,0.92}

\lstdefinestyle{mystyle}{
    backgroundcolor=\color{backcolour},   
    commentstyle=\color{codegreen},
    keywordstyle=\color{magenta},
    numberstyle=\tiny\color{codegray},
    stringstyle=\color{codepurple},
    basicstyle=\BeraMonottfamily\footnotesize,
    breakatwhitespace=false,         
    breaklines=true,                 
    captionpos=b,                    
    keepspaces=true,                 
    numbers=left,                    
    numbersep=5pt,                  
    showspaces=false,                
    showstringspaces=false,
    showtabs=false,                  
    tabsize=2
}
\lstset{style=mystyle}

\title{CS249 Fall 2020\\
       Problem Set 1: Statistical Inference}
\author{Christopher Munoz Cortes}
\date{October 26, 2020}

\usepackage{natbib}
\usepackage{graphicx}

\begin{document}

\maketitle

\section{Maximum Likelihood Estimation}
\begin{enumerate}[label={(\alph*)}]
    \item Assume we have $n$ positive data points $Y_1,...,Y_n \sim$
    Exponential$(\theta)$. Compute the maximum likelihood estimator for $\theta$.
    
    We know that the probability density function of an exponential 
    distribution is given by
    \begin{align*}
        f(y;\theta) =
        \begin{cases}
            \theta e^{-\theta y} & y \geq 0 \\
            0 & \text{o.w.}
        \end{cases}
    \end{align*}
    The likelihood function of this distribution is
    \[
        \mathcal{L}(\theta) = \prod_{i=1}^{n} \theta e^{-\theta y_i} = 
        \left(\theta e^{-\theta y_1}\right) \left(\theta e^{-\theta  
        y_2}\right) ...\left(\theta e^{-\theta y_n}\right) =
        \theta^n e^{-\theta\sum_{i=1}^{n}y_i}
    \]
    The MLE of this exponential distribution can be calculated taking the 
    derivative of the log-likelihood function (which is easier to compute 
    than the likelihood function) and setting it equal to zero.
    \begin{align*}
        \dfrac{d}{d\theta}\left[ln{\mathcal{L}(\theta)}\right]  &= 0 \\
        \dfrac{d}{d\theta} \left[\ln{\left(\theta^n e^{-\theta \sum_{i=1}^{n}
        y_i}\right)}\right]  &= 0 \\
        \dfrac{d}{d\theta}\left[n\ln{\theta} - \theta \sum_{i=1}^{n} 
        y_i\right] &= 0 \\
        \dfrac{n}{\theta} - \sum_{i=1}^{n}y_i &= 0
    \end{align*}
    Solving for the $\theta$:
    \begin{equation*}
        \boxed{\theta = \dfrac{n}{\sum_{i=1}^{n} y_i}}
    \end{equation*}
    \item Assume we have $n$ positive data points $Y_1,...,Y_n \sim$ 
    Uniform$(0, \theta)$, meaning the data is coming from a uniform distribution in the 
    interval $[0,\theta]$. Compute the maximum likelihood estimator for $\theta$.
    
    The pdf of $Y \sim$ Uniform$(0,\theta)$ is
    \begin{equation}
        f(y;\theta) = 
        \begin{cases}
            \frac{1}{\theta} & 0 \leq y \leq \theta \\
            0 & \text{o.w.}
        \end{cases}
    \end{equation}
    Now consider a fixed value of $\theta$. If $Y_i > \theta$ for some $i$, then 
    $f(Y_i;\theta)$ = 0 and therefore $\mathcal{L}(\theta)$ = 0. This means that if any 
    $Y_i > 0$, $\mathcal{L}(\theta) = 0$. This can be written as $\mathcal{L}(\theta) = 0$, if $\theta < Y_n$, where $Y_{(n)} = \max\{Y_1,\ldots,Y_n\}$.
    On the other hand, if $Y_{(n)} \leq \theta$, $f(y;\theta) = 1/\theta$ and consequently $\mathcal{L}(\theta) = 1/\theta^n$
    Hence,
    \begin{equation}
            \mathcal{L}(\theta) =
        \begin{cases}
           \frac{1}{\theta^n} & \theta \geq Y_{(n)} \\
           0 & \theta < Y_{(n)}
        \end{cases}
    \end{equation}
    Finally, since $\mathcal{L}(\theta)$ is strictly decreasing in $[Y_{(n)}, \infty)$, the maximum value occurs at $Y_{(n)}$ and $\hat{\theta} = Y_{(n)}$.
    
    \item Estimate bias, variance, and RMSE for the estimator in (b) when $\theta = 10$ 
    and $n=100$ by doing simulations in Python.
    \begin{lstlisting}[language=Python, caption=Code for part (c)]
import numpy as np
np.random.seed(0)

# Estimate bias, variance, and RMSE for a estimator from part (b)
# for theta=10, n=100
low = 0
theta = 10
n=100
theta_hats = []

def question_one(low, theta, n):
  theta_hats = []
  for i in range(1000):
    y = np.random.uniform(low=0, high=theta, size=n)
    theta_hat = np.max(y)
    theta_hats.append(theta_hat)
  theta_hats = np.array(theta_hats)

  # Calculate bias, variance, and RMSE of the estimator theta_hat
  # To calculate the bias, take the mean of the theta_hats as the expectation
  bias = np.mean(theta_hats) - theta
  variance = np.var(theta_hats)
  mse = bias**2 + variance
  rmse = np.sqrt(mse)
  return bias, variance, rmse

bias, variance, rmse = question_one(low, theta, n)
\end{lstlisting}

Output:
    \begin{itemize}
        \item bias$(\hat{\theta})$: -0.100
        \item Var$(\hat{\theta})$: 0.00829
        \item RMSE$(\hat{\theta})$: 0.135
    \end{itemize}
    
    \item What are the estimated values in (c) if $n$ increases to 500? Describe your observations.
    If $n$ increases from 100 to 500, $\hat{\theta}$ starts to converge towards $\theta$ and the ``spread'' of the distribution begins to shrink, as evidenced by the decreasing variance and RSME. In other words, $\hat{\theta}$ becomes a better estimator of $\theta$.
    \begin{itemize}
        \item bias$(\hat{\theta})$: -0.0193
        \item Var$(\hat{\theta})$: 0.000360
        \item RMSE$(\hat{\theta})$: 0.0271
    \end{itemize}
\end{enumerate}
\pagebreak

\section{Bootstrap}
\begin{enumerate}[label={(\alph*)}]
    \item Assume we have a sample of 20 IDD data points with the following values:
    
    3.0 1.9 6.4 5.9 4.2 6.2 1.4 2.9 2.3 4.8 7.8 4.5
    0.7 4.4 4.4 6.5 7.6 6.1 2.7 1.6
    
    Assume we define $T$ as the median among 20 data points. Use bootstrap
    to estimate the standard error and the confidence interval for $T$.
    \begin{itemize}
        \item Bootstrap standard error: 0.733
        \item Bootstrap 95\% confidence interval: (2.964, 5.836)
    \end{itemize}
    \begin{lstlisting}[language=Python, caption=Standard Error and Confidence Interval for $T$]
# Create array with data
data = np.array([3.0, 1.9, 6.4, 5.9, 4.2, 6.2, 1.4, 2.9, 2.3, 4.8, 7.8, 4.5,
                 0.7, 4.4, 4.4, 6.5, 7.6, 6.1, 2.7, 1.6])
                 
# Create the bootstrap datasets and calculate the median for each
t_boot_list = [np.median(np.random.choice(data, len(data), replace=True))
               for _ in range(1000)]

# Calculate the standard error and the confidence interval
t_boot_se = np.std(t_boot_list)
ci_lower = np.median(data) - 1.96*t_boot_se
ci_upper = np.median(data) + 1.96*t_boot_se                 
    \end{lstlisting}
    
    \item Use \lstinline{y = np.random.normal(0, 5, 100)} to generate 100 data points from
    the normal distribution $N(0,5)$. Consider the generated data points as your observed data.
        \begin{itemize}
            \item Apply the bootstrap method to estimate the standard error for
            $T_1$ and $T_2$, where $T_1$ is the sample median and $T_2$ is the maximum
            value in the sample.
            \begin{itemize}
                \item $T_1$ Bootstrap standard error: 0.504
                \item $T_2$ Bootstrap standard error: 1.680
            \end{itemize}
            
            \item Next, compute the actual standard error for $T_1$ and $T_2$ by simulating
            many times from the data source (i.e., $N(0, 5)$).
            \begin{itemize}
                \item Actual standard error by sim from source for $T_1$: 0.623
                \item Actual standard error by sim from source for $T_2$: 2.196
            \end{itemize}
        \end{itemize}
\begin{lstlisting}[language=Python, caption=Bootstrap and Actual Standard Error]
# Generate new data points. This is the observed data.
y = np.random.normal(0,5,100)

# Apply the bootstrap method to estimate
# T1: sample median
# T2: max value
t1_boot_list = [np.median(np.random.choice(y, len(y), replace=True))
               for _ in range(1000)]
t1_boot_se = np.std(t1_boot_list)

t2_boot_list = [np.max(np.random.choice(y, len(y), replace=True))
                for _ in range(1000)]
t2_boot_se = np.std(t2_boot_list)

# Compute the actual standard error for T1 and T2 by simulating many times
# from the data source
sim_t1_list = [np.median(np.random.normal(0,5,100)) for _ in range(1000)]
sim_t1_se = np.std(sim_t1_list)

sim_t2_list = [np.max(np.random.normal(0,5,100)) for _ in range(1000)]
sim_t2_se = np.std(sim_t2_list)
\end{lstlisting}

\end{enumerate}
\pagebreak

\section{Parametric Bootstrap}
\begin{enumerate}[label={(\alph*)}]
    \item First, we need a parametric distribution model for the data. Let’s
    assume the data points in part (a) are generated from a normal distribution
    with the mean of $\theta$ and the standard deviation of 2. Using
    the observed data, compute $\hat{\theta}$, the estimated value of the $\theta$ in this
    distribution model.
    
    We can compute $\hat{\theta}$ using MLE:
    \begin{align*}
        \mathcal{L}(\theta,\sigma) &= \prod_i
        \dfrac{1}{\sqrt{2\pi\sigma^2}}\exp{\left\{-\dfrac{1}{2\sigma^2}
        \left(X_i - \theta \right)^2\right\}} \\
        \mathcal{L}(\theta,\sigma) &= 
        \dfrac{1}{\sqrt{(2\pi\sigma^2)^n}}\exp{\left\{-\dfrac{1}{2\sigma^2}
        \sum_i{\left(X_i - \theta \right)^2} \right\}}
    \end{align*}
    Taking the log on both sides,
    \begin{align*}
        \ln\mathcal{L}(\theta,\sigma) &= -\dfrac{n}{2}\ln 2\pi\sigma^2 - 
        \dfrac{1}{2\sigma^2} \sum_i{(X_i - \theta)^2}
    \end{align*}
    Taking the partial derivative with respect to $\theta$,
    \begin{equation*}
        \dfrac{\partial}{\partial \theta} \ln \mathcal{L}
        (\theta, \sigma) =
        \dfrac{1}{\sigma^2} \sum_i (X_i - \theta) =
        \dfrac{1}{\sigma^2} n(\bar{x} - \theta) \\
    \end{equation*}
    Setting equal to 0 and solving for $\theta$,
    \begin{equation*}
        \boxed{\hat{\theta} = \bar{x}}
    \end{equation*} 
    Consequently, $\boxed{\hat{\theta} = \bar{x} = 4.265}$.
    
    \item Generate bootstrap samples (20 data points each) from the distribution
    with the estimated parameter (i.e., $N(\hat{\theta},2)$).
    
    \texttt{See listing below for code}
    
    \item In each bootstrap sample, estimate the value of $\theta$ again based on the
    simulated data. 
    
    \texttt{See listing below for code}
    
    \item Compute the standard deviation for the estimated parameter among
    the bootstrap samples. This standard deviation is the estimated standard
    error for $\theta$. 
    
    The estimated standard error for $\theta$ was 0.453.
    
\begin{lstlisting}[language=Python, caption=Parametric Bootstrap code]
# Compute theta_hat for this distribution model (using MLE)
theta_hat = np.mean(data)
print(f"Estimator for the mean of theta (theta_hat): {theta_hat}")

# Generate bootstrap samples, i.e. simulate the data from source
sim_data = [np.random.normal(theta_hat,2,100) for _ in range(1000)]

# Estimate the value of theta_hat in each bootstrap sample
theta_hats = [np.mean(sim_sample) for sim_sample in sim_data]

# Compute the standard error fot theta_hat
theta_hat_se = np.std(theta_hats)
print(f"Standard error for theta_hat: {theta_hat_se}")
\end{lstlisting}
\end{enumerate}
\pagebreak

\section{Bayesian Inference}
\begin{enumerate}[label={(\alph*)}]
    \item Assume we have observed IID data points $X_1,\ldots,X_{10}$ from the distribution
    $N(\theta,1)$, where the sample average, $\bar{X}$, is 1.68. If our prior
    belief about $\theta$ can be described by $N(0,3)$, compute the posterior distribution
    of $\theta$ after observing this data.
%    \begin{table}[h]
%        \centering
%        \begin{tabular}{@{}cccc@{}}
%        \toprule
%        Parameter & \texttt{mean} & \texttt{se\_mean} & \texttt{sd} \\
%        \midrule
%        $\theta$ & 1.33 & 0.01 & 0.39 \\
%        $\sigma$ & 1.24 & 0.01 & 0.35 \\ \bottomrule
%        \end{tabular}
%    \end{table}
    We know that the posterior distribution is given by
    \begin{align*}
        f(\theta|X_1,\ldots,X_{10}) \propto \mathcal{L}(\theta)f(\theta)
    \end{align*}
    \begin{align*}
        \mathcal{L}(\theta)f(\theta) &= \prod_{i=1}^{n} \dfrac{1}{\sqrt{2\pi\sigma^2}}
        \exp{\left\{-\dfrac{(X_i - \theta)^2}{2\sigma^2}\right\}}
        \dfrac{1}{\sqrt{2\pi\sigma_0^2}} \exp{\left\{ - \dfrac{(\theta -
        \theta_0)^2}{2\sigma_0^2}\right\}} \\
        &= \dfrac{1}{\sqrt{(2\pi\sigma^2)^n}}\exp{\left\{-\dfrac{1}{2\sigma^2}
        \sum_i{\left(X_i - \theta \right)^2} \right\}}
        \dfrac{1}{\sqrt{2\pi\sigma_0^2}} \exp{\left\{ - \dfrac{(\theta -
        \theta_0)^2}{2\sigma_0^2}\right\}} \\
        &= \dfrac{1}{(2\pi)^\frac{n+1}{2} \sqrt{\sigma^{2n} \sigma_0^2}}
        \exp{\left\{ \dfrac{\theta^2 + 2\theta \theta_0 - \theta_0^2}{2\sigma_0^2}
        -\sum_{i=1}^n \dfrac{X_i^2 2\theta X_i + \theta^2}{2\sigma^2}\right\}} \\
        &\propto \exp{\left\{ \dfrac{-\theta^2\sigma^2 + 2\theta\theta_0\sigma^2
        - \theta_0^2\sigma^2 - (\sum_i X_i^2\sigma_0^2 - -2\sum_i X_i\theta\sigma_0^2 + 
        n\theta^2\sigma_0^2)}{2\sigma_0^2\sigma^2}\right\}} \\ 
        &\propto \exp{\left\{ \dfrac{-\theta^2(\sigma^2 + n\sigma_0^2) +2\theta(\theta_0 \sigma^2
        + \sum_i X_i\sigma_0^2) - (\theta_0^2\sigma^2 + \sum_i X_i^2 \sigma_0^2)} {
        2\sigma_0^2 \sigma^2} \right\}} \\
        \begin{split}
            &\propto \exp{\left\{ \dfrac{-\theta^2 + 2\theta \left(\dfrac{\theta_0\sigma^2 +
            \sum_i X_i\sigma_0^2}{\sigma^2 + n\sigma_0^2}\right) - 
            \left(\dfrac{\theta_0^2\sigma^2 + \sum_i X_i^2 \sigma_0^2}{\sigma^2 + 
            n\sigma_0^2}\right)^2}{\dfrac{2\sigma_0^2\sigma^2}{\sigma^2 + 
            n\sigma_0^2}}\right\}} \\
            &\qquad\qquad \times
            \exp{\left\{-\dfrac{\theta_0^2\sigma^2 + \sum_i X_i^2 \sigma_0^2}
            {2\sigma_0^2\sigma^ r}\right\}}
        \end{split}\\
        &\propto \exp{\left\{-\dfrac{\left( \theta - 
        \dfrac{\theta_0^2\sigma^2 + \sum_i X_i^2 \sigma_0^2}{\sigma^2 
        + n\sigma_0^2}\right)^2}{2\dfrac{\sigma_0^2 \sigma^2}{\sigma^2 + 
        n\sigma_0^2}}\right\}}
    \end{align*}
    Here we can define:
    \[
        \sigma_1^2 = \dfrac{\sigma_o^2 \sigma^2}{\sigma^2 + n\sigma_0^2} =
        \dfrac{1}{\frac{\sigma^2 + n\sigma_0^2}{\sigma_0^2\sigma^2}} =
        \dfrac{1}{\frac{1}{\sigma_0^2} + \frac{1}{\frac{\sigma^2}{n}}} =
        \left( \dfrac{1}{\sigma_0^2} + \dfrac{1}{\frac{\sigma^2}{n}} \right)^{-1}
    \]
    and
    \[
        \theta_1 = \dfrac{\theta_0\sigma^2 + \sum_i X_i \sigma_0^2}{\sigma^2 + n\sigma_0^2} =
        \dfrac{\theta_0\sigma_0^{-2} + \sum_i X_i \sigma^{-2}}{\sigma_0^{-2} + n\sigma^{-2}} =
        \sigma_1^2 \left(\theta_0 \sigma_0^{-2} + \sum_i X_i \sigma^{-2}\right) =
        \sigma_1^2 \left(\dfrac{\theta_0}{\sigma_0^2} + \dfrac{\bar{X}}{\frac{\sigma^2}{n}}\right)
    \]
    which shows that the resulting expression for the posterior distribution corresponds to a 
    normal distribution with mean $\theta_1$ and standard deviation $\sigma_1$, i.e. 
    $N(\theta_1,\sigma_1)$.
    
    Plugging in our values, we have:
    \[
        \sigma_1^2 = \dfrac{\sigma_0^2\sigma^2}{\sigma^2 + n\sigma_0^2} = 
        \dfrac{(3)^2(1)^2}{(1)^2 + 10(3)^2} = 0.0989 
        \rightarrow \boxed{\sigma_1 = 0.314}
    \]
    and
    \[
        \theta_1 = \sigma_1^2 \left(\dfrac{\theta_0}{\sigma_0^2} + 
        \dfrac{\bar{X}}{\frac{\sigma^2}{n}}\right) =
        0.0989 \left(\dfrac{0}{3^2} + \dfrac{1.68}{1/10}\right) 
        \rightarrow \boxed{\theta_1 = 1.66}
    \]
    \item Assume we have gathered more evidence about $\theta$ in part (a) by 
    experimenting with another signal related to the same source and gathering new data 
    points $Z_1, \ldots, Z_{20} \overset{\text{\emph{iid}}}{\sim} N(\theta,4)$. If 
    $\bar{Z} = 0.8$, 
    compute the new posterior distribution of $\theta$ after seeing both 
    samples. Assume samples in part (a) and (b) are independent from each 
    other.
    
    Considering the prior distribution as $N(\theta_1, \sigma_1) = N(1.66,0.314)$, the 
    posterior distribution given the new data $Z_1, \ldots, Z_{20} \sim N(\theta,4)$ is:
    \[
        \sigma_2^2 = \dfrac{\sigma_1^2\sigma^2}{\sigma^2 + n\sigma_1^2} = 
        \dfrac{(0.0989)(4)^2}{(4)^2 + 20(0.0989)} = 0.088 
        \rightarrow \boxed{\sigma_2 = 0.297}
    \]
    and
    \[
        \theta_2 = \sigma_2^2 \left(\dfrac{\theta_1}{\sigma_1^2} + \dfrac{\bar{X}}{\frac{\sigma^2}{n}}\right) =
        0.088 \left(\dfrac{1.66}{0.0989} + \dfrac{0.8}{4^2/20}\right) 
        \rightarrow \boxed{\theta_2= 1.565} 
    \]
    \item How does your inference result change if your sample in part (b) had
    more data points?
    
    As the number of data points $n$ increases, the posterior distribution becomes more influenced by the likelihood function of the observed data. Eventually, for very large values of $n$, the prior distribution vanishes and the Bayesian inference method yield a similar result to a frequentist approach.
    \item How does your inference result change if you had more certainty about
    your prior belief (i.e. the prior distribution had lower variance)?
    
    The posterior distribution would be closer to the prior distribution, provided that
    the observed data didn't indicate a significantly different distribution. 
    
%\begin{lstlisting}[language=Python, caption=Bayesian Inference Code]
%import pystan
%
%n = 10
%x_bar = 1.68
%y = np.random.normal(x_bar, 1, n)
%
%model_code = """
%data {
%  int<lower=0> n;
%  real y[n];
%}
%parameters {
%  real<lower=0,upper=100> theta;
%  real<lower=0,upper=10> sigma;
%}
%model {
%  y ~ normal(theta,sigma);
%}
%"""
%norm_dat = {
%    'n': 10,
%    'y': y
%}
%
%# Fit the model with the original data
%model = pystan.StanModel(model_code=model_code)
%fit = model.sampling(data=norm_dat, iter=1000, chains=4, n_jobs=2)
%print(fit)
%
%# Fit the model with the new data
%new_dat = {
%    'n': 20,
%    'y': np.random.normal(0.8, 4, 20),
%}
%
%fit2 = model.sampling(data=new_dat)
%print(fit2)
%\end{lstlisting}
    
\end{enumerate}

%\bibliographystyle{plain}
%\bibliography{references}
\end{document}
